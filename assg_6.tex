\def\mytitle{MATRICES USING PYTHON}
\def\myauthor{THOUTU RAHUL RAJ}
\def\contact{rdj4648@gmail.com}
\def\mymodule{Future Wireless Communication (FWC)}
\documentclass[10pt, a4paper]{article}
\usepackage[a4paper,outer=1.5cm,inner=1.5cm,top=1.75cm,bottom=1.5cm]{geometry}
\twocolumn
\usepackage{graphicx}
\graphicspath{{./images/}}
\usepackage[colorlinks,linkcolor={black},citecolor={blue!80!black},urlcolor={blue!80!black}]{hyperref}
\usepackage[parfill]{parskip}
\usepackage{lmodern}
\usepackage{amsmath,amsfonts,amssymb,amsthm}
\usepackage{tikz}
	\usepackage{physics}
%\documentclass[tikz, border=2mm]{standalone}
\usepackage{karnaugh-map}
%\documentclass{article}
\usepackage{tabularx}
\usepackage{circuitikz}
\usetikzlibrary{calc}
\usepackage{amsmath}
\usepackage{amssymb}
\renewcommand*\familydefault{\sfdefault}
\usepackage{watermark}
\usepackage{lipsum}
\usepackage{xcolor}
\usepackage{listings}
\usepackage{float}
\usepackage{titlesec}
\providecommand{\norm}[1]{\left\lVert#1\right\rVert}
\providecommand{\sbrak}[1]{\ensuremath{{}\left[#1\right]}}
\providecommand{\lsbrak}[1]{\ensuremath{{}\left[#1\right.}}
\providecommand{\rsbrak}[1]{\ensuremath{{}\left.#1\right]}}
\providecommand{\brak}[1]{\ensuremath{\left(#1\right)}}
\providecommand{\lbrak}[1]{\ensuremath{\left(#1\right.}}
\providecommand{\rbrak}[1]{\ensuremath{\left.#1\right)}}
\providecommand{\cbrak}[1]{\ensuremath{\left\{#1\right\}}}
\providecommand{\lcbrak}[1]{\ensuremath{\left\{#1\right.}}
\providecommand{\rcbrak}[1]{\ensuremath{\left.#1\right\}}}
\newcommand{\myvec}[1]{\ensuremath{\begin{pmatrix}#1\end{pmatrix}}}
\let\vec\mathbf
\providecommand{\mtx}[1]{\mathbf{#1}}
\titlespacing{\subsection}{1pt}{\parskip}{3pt}
\titlespacing{\subsubsection}{0pt}{\parskip}{-\parskip}
\titlespacing{\paragraph}{0pt}{\parskip}{\parskip}
\newcommand{\figuremacro}[5]

\begin{document}

\title{\mytitle}
\author{\myauthor\hspace{1em}\\\contact\\FWC22008\hspace{6.5em}IITH\hspace{0.5em}\mymodule\hspace{6em}ASSIGN-4}
\date{}
	\maketitle
		
	\tableofcontents
\vspace{5mm}
   \section{Problem}
\textbf{Find the equation of circle passing through the points (2,3) and (-1,1) and whose centre is on the line x-3y-11=0.}
 \section{Construction}
 	\begin{center}
     Figure of Construction
     \includegraphics[scale=0.5]{fig.pdf} 
  	\end{center}

   \section{Solution}


\vspace{.25 cm}
\textbf{To Find:}
 The equation of circle.\\
\textbf{Given}, points passing through circle (2,3) and (-1,1), And equation of line passing through the center of circle x-3y-11=0. 

We know that 
\begin{align}
\vec{x}^{\top}\vec{V}\vec{x}+2\vec{u}^{\top}\vec{x}+f=0
\end{align}	
$\vec{V}$ = $\begin{pmatrix}
 1 & 0\\
 0 & 1
 \end{pmatrix}$,
 
 By substituting the given 2 points  
  $\vec{x1}$=$\vec{\begin{pmatrix}2 \\3 \end{pmatrix}}$
 and
 $\vec{x2}$=$\vec{\begin{pmatrix}-1 \\1 \end{pmatrix}}$\\
 we get one equation in the form of $\vec{u}$\\
 
 and, if we substitute x1,x2 value in the given line equation we get\\
\begin{align}
13+\vec{\begin{pmatrix}2 &  3 \end{pmatrix}}u+f =0
\end{align}
 \begin{align}
2+\vec{\begin{pmatrix}-2 & 2 \end{pmatrix}}u+f =0
\end{align}
By solving equations 2 and 3 we get
 \begin{align}
\vec{\begin{pmatrix}8 & 4 \end{pmatrix}}u =-11
\end{align}
From the given line equation passing through the center
 \begin{align}
\vec{\begin{pmatrix}1 & -3 \end{pmatrix}}u=-11
\end{align}
by solving equation 4 and 5 we get center as
 $\vec{u}$=$\vec{\begin{pmatrix}7/2 \\-5/2 \end{pmatrix}}$
By substituting value of u in equation 2 we get\\ 
\begin{align}
f=10
\end{align}
from u and f we can find radius
\begin{align}
 r = \sqrt{\norm{\vec{(u)}}^2-f} 
\end{align}
\begin{align}
r = \sqrt{65/4}
\end{align}

And, from them we can find the equation of circle.\\
 \begin{align}
\vec{x}^{\top}\vec{V}\vec{x}+2\vec{u}^{\top}\vec{x}+f=0
\end{align}	
$\vec{V}$ = $\begin{pmatrix}
 1 & 0\\
 0 & 1
 \end{pmatrix}$, 
  $\vec{u}$=$\vec{\begin{pmatrix}7/2 \\-5/2 \end{pmatrix}}$
  f = 10
steps for constructing above figure are:  
\begin{center}
\begin{tabular}{|c|c|c|}
	\hline
	\textbf{Symbol}&\textbf{Value}&\textbf{Description}\\
	\hline
	$r$&$\sqrt{65/4}$&Radius of the circle\\
	\hline
	\textbf{C}&$\
	\begin{pmatrix}
		7/2 \\
		-5/2 \\
	\end{pmatrix}$
	&center of circle\\
	\hline
\end{tabular}
\end{center}
\vspace{1mm}
\textbf{Termux commands :}
\begin{lstlisting}
bash r.sh
\end{lstlisting}
\bibliographystyle{ieeetr}
\end{document}
